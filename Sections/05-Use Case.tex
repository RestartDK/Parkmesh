
\subsection{Use Case}

Consider Sarah (fictional character), a marketing professional who works in downtown Chicago. Every weekday morning, she drives to her office building which has a 300-space parking garage that's typically 90\% full by 9:00 AM. Without our solution, Sarah faces three alternatives:

\begin{enumerate}[leftmargin=*]
  \item Arrive 30 minutes early to guarantee a spot, wasting valuable personal time daily
  \item Pay \$25/day for valet parking or reserved spots, adding up to \$6,000 annually
  \item Circle the garage for 5-15 minutes searching for an available space, burning fuel and increasing stress before important meetings
\end{enumerate}

With our decentralized vehicle sensor network, her phone immediately directs her to level 3, where two cars equipped with our sensors have detected an available space between them. She parks in under a minute and arrives at her desk calm and prepared.

This solution addresses not just the immediate problem of finding parking, but the cascade of negative effects: the wasted fuel (\$300-500 annually per driver), increased stress affecting workplace performance, the unpredictability that forces inefficient time buffers in schedules, and the environmental impact of unnecessary emissions.

Unlike existing alternatives like parking apps that rely on historical predictions or garage-installed sensor systems that cost \$300-500 per space \footcite{ParkingLogix2024} to implement, our vehicle-based approach provides real-time certainty without the massive infrastructure investment. As a side-effect building owners will benefit as the system increases tenant satisfaction without renovation costs. For drivers like Sarah, it eliminates a daily pain point that affects both their wallet and well-being.